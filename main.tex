% Options for packages loaded elsewhere
\PassOptionsToPackage{unicode}{hyperref}
\PassOptionsToPackage{hyphens}{url}
%
\documentclass[
]{article}
\usepackage{amsmath,amssymb}
\usepackage{lmodern}
\usepackage{iftex}
\ifPDFTeX
  \usepackage[T1]{fontenc}
  \usepackage[utf8]{inputenc}
  \usepackage{textcomp} % provide euro and other symbols
\else % if luatex or xetex
  \usepackage{unicode-math}
  \defaultfontfeatures{Scale=MatchLowercase}
  \defaultfontfeatures[\rmfamily]{Ligatures=TeX,Scale=1}
\fi
% Use upquote if available, for straight quotes in verbatim environments
\IfFileExists{upquote.sty}{\usepackage{upquote}}{}
\IfFileExists{microtype.sty}{% use microtype if available
  \usepackage[]{microtype}
  \UseMicrotypeSet[protrusion]{basicmath} % disable protrusion for tt fonts
}{}
\makeatletter
\@ifundefined{KOMAClassName}{% if non-KOMA class
  \IfFileExists{parskip.sty}{%
    \usepackage{parskip}
  }{% else
    \setlength{\parindent}{0pt}
    \setlength{\parskip}{6pt plus 2pt minus 1pt}}
}{% if KOMA class
  \KOMAoptions{parskip=half}}
\makeatother
\usepackage{xcolor}
\usepackage{longtable,booktabs,array}
\usepackage{calc} % for calculating minipage widths
% Correct order of tables after \paragraph or \subparagraph
\usepackage{etoolbox}
\makeatletter
\patchcmd\longtable{\par}{\if@noskipsec\mbox{}\fi\par}{}{}
\makeatother
% Allow footnotes in longtable head/foot
\IfFileExists{footnotehyper.sty}{\usepackage{footnotehyper}}{\usepackage{footnote}}
\makesavenoteenv{longtable}
\usepackage{graphicx}
\makeatletter
\def\maxwidth{\ifdim\Gin@nat@width>\linewidth\linewidth\else\Gin@nat@width\fi}
\def\maxheight{\ifdim\Gin@nat@height>\textheight\textheight\else\Gin@nat@height\fi}
\makeatother
% Scale images if necessary, so that they will not overflow the page
% margins by default, and it is still possible to overwrite the defaults
% using explicit options in \includegraphics[width, height, ...]{}
\setkeys{Gin}{width=\maxwidth,height=\maxheight,keepaspectratio}
% Set default figure placement to htbp
\makeatletter
\def\fps@figure{htbp}
\makeatother
\usepackage[normalem]{ulem}
\setlength{\emergencystretch}{3em} % prevent overfull lines
\providecommand{\tightlist}{%
  \setlength{\itemsep}{0pt}\setlength{\parskip}{0pt}}
\setcounter{secnumdepth}{-\maxdimen} % remove section numbering
\ifLuaTeX
  \usepackage{selnolig}  % disable illegal ligatures
\fi
\IfFileExists{bookmark.sty}{\usepackage{bookmark}}{\usepackage{hyperref}}
\IfFileExists{xurl.sty}{\usepackage{xurl}}{} % add URL line breaks if available
\urlstyle{same} % disable monospaced font for URLs
\hypersetup{
  hidelinks,
  pdfcreator={LaTeX via pandoc}}

\author{}
\date{}

\begin{document}

Этот файлик для региона, в будущем сделаю его и для всероса, добавлю
интерактивки, двойной запуск и открытые тесты. WL Лёха

\hypertarget{ux43eux431ux449ux430ux44f-ux438ux43dux444ux43eux440ux43cux430ux446ux438ux44f}{%
\section{Общая
информация}\label{ux43eux431ux449ux430ux44f-ux438ux43dux444ux43eux440ux43cux430ux446ux438ux44f}}

Основные выводы из памятки участника регионального этапа:

\begin{enumerate}
\def\labelenumi{\arabic{enumi}.}
\item
  у вас всего 50 попыток на одну задачу, 200 суммарно по всем задачам.
  Раньше было ограничение 150 попыток суммарно;
\item
  берется лучшее решение (склейки нету, если первое решение заходит на
  одну подгруппу, а второе на другую, то заифайте и объедините решения);
\item
  из системы нельзя выгружать посылки, поэтому сохраняйте все решения
  (например, название файла - номер посылки в системе);
\item
  в задачах возможна потестовая оценка.
\end{enumerate}

\hypertarget{ux43dux435-ux437ux430ux445ux43eux434ux438ux442-ux43fux43e-pe-presentation-error}{%
\section{Не заходит по PE (presentation
error)}\label{ux43dux435-ux437ux430ux445ux43eux434ux438ux442-ux43fux43e-pe-presentation-error}}

PE на реге 21 веке. Что за кринж? Да, на регионе может быть PE, но это
не значит, что если вы выведете в неправильном формате, то у вас будет
PE, может и WA.

В памятке участника как пример возникновения PE: ``Программа должна
вывести числа в одной строке через пробел, а вывела их в разных строках
или наоборот``. Современные чекеры не будут ругаться на такие случаи, но
все же выводите ответ, как просит задача.

Бывает два вида чекеров (проверяющих ответ участника программ):

\begin{enumerate}
\def\labelenumi{\arabic{enumi}.}
\item
  Стандартный (проверяет, равен ли ответ участника ответу жюри). В таком
  случае PE можно получить только если выводить совсем не то, например
  строку вместо числа.
\item
  Кастомный (когда верных ответов несколько и нельзя просто сверять с
  ответом жюри). В таком случае PE совсем нельзя получить, всегда будете
  получать WA.
\end{enumerate}

В целом нужно воспринимать PE как WA, скорее всего оно не имеет смысла.

\hypertarget{ux43dux435-ux437ux430ux445ux43eux434ux438ux442-ux43fux43e-wa-wrong-answer}{%
\section{Не заходит по WA (wrong
answer)}\label{ux43dux435-ux437ux430ux445ux43eux434ux438ux442-ux43fux43e-wa-wrong-answer}}

\begin{enumerate}
\def\labelenumi{\arabic{enumi})}
\item
  Забыли убрать отладочный вывод.
\item
  Когда используете double, может вывести nan или inf (возможно выведет,
  как PE).
\item
  Переполнение int.
\item
  Undefined behavior (поведение зависит от компилятора, бывает,
  например, при выходе за границы стат. массива).
\item
  Неверная идея или баг в коде (ну тут мои полномочия всё).
\end{enumerate}

\hypertarget{ux43dux435-ux437ux430ux445ux43eux434ux438ux442-ux43fux43e-tl-time-limit}{%
\section{Не заходит по TL (time
limit):}\label{ux43dux435-ux437ux430ux445ux43eux434ux438ux442-ux43fux43e-tl-time-limit}}

На полигоне (g++ 14) одна секунда - это 2е9 ifов (потому что прагма O2
автоматически включена на CF), 2e8 операций деления и 5е8 операций
остатка. На моем компьютере (g++ 17) одна секунда - это 7e8 операций
сложения, 3e8 операций деления, 5e8 ifов, 4e8 остатка. И появляется
резонный вопрос, чему равна одна секунда в c++? Невозможно понять из-за
константы операций, однако я думаю безопасно считать, что это примерно
от 1e8 до 3e8 действий, вы сможете сделать больше если захотите.

Однако создатели задач региона и всероса очень редко планируют, чтобы вы
их запихивали (либо дают за это мало баллов). Поэтому:

\begin{longtable}[]{@{}
  >{\raggedright\arraybackslash}p{(\columnwidth - 4\tabcolsep) * \real{0.3333}}
  >{\raggedright\arraybackslash}p{(\columnwidth - 4\tabcolsep) * \real{0.3333}}
  >{\raggedright\arraybackslash}p{(\columnwidth - 4\tabcolsep) * \real{0.3333}}@{}}
\toprule()
\begin{minipage}[b]{\linewidth}\raggedright
Ограничения по n
\end{minipage} & \begin{minipage}[b]{\linewidth}\raggedright
Авторская асимптотика
\end{minipage} & \begin{minipage}[b]{\linewidth}\raggedright
Вы можете упихать (я в вас верю \textgreater-\textless{} )
\end{minipage} \\
\begin{minipage}[b]{\linewidth}\raggedright
до 10
\end{minipage} & \begin{minipage}[b]{\linewidth}\raggedright
O(n*n!)
\end{minipage} & \begin{minipage}[b]{\linewidth}\raggedright
O(n\^{}2*n!) или O(n\^{}8)
\end{minipage} \\
\begin{minipage}[b]{\linewidth}\raggedright
до 15-18
\end{minipage} & \begin{minipage}[b]{\linewidth}\raggedright
O(3\^{}n)
\end{minipage} & \begin{minipage}[b]{\linewidth}\raggedright
O(3\^{}n)
\end{minipage} \\
\begin{minipage}[b]{\linewidth}\raggedright
до 20
\end{minipage} & \begin{minipage}[b]{\linewidth}\raggedright
O(2\^{}n) или O(2\^{}n*n)
\end{minipage} & \begin{minipage}[b]{\linewidth}\raggedright
O(2\^{}n*n\^{}2)
\end{minipage} \\
\begin{minipage}[b]{\linewidth}\raggedright
до
40-50(\href{https://neerc.ifmo.ru/wiki/index.php?title=Meet-in-the-middle}{\uline{MITM}})
\end{minipage} & \begin{minipage}[b]{\linewidth}\raggedright
O(2\^{}(n/2)) или O(2\^{}(n/2)*n)
\end{minipage} & \begin{minipage}[b]{\linewidth}\raggedright
То же самое, но с большой константой
\end{minipage} \\
\begin{minipage}[b]{\linewidth}\raggedright
до 1000
\end{minipage} & \begin{minipage}[b]{\linewidth}\raggedright
O(n\^{}2) или O(n\^{}2 * log(n))
\end{minipage} & \begin{minipage}[b]{\linewidth}\raggedright
O(n\^{}2 * log\^{}2(n))
\end{minipage} \\
\begin{minipage}[b]{\linewidth}\raggedright
до 1e5
\end{minipage} & \begin{minipage}[b]{\linewidth}\raggedright
O(n), O(n*log(n)), O(n*log\^{}2(n)) или O(n*sqrt(n))
\end{minipage} & \begin{minipage}[b]{\linewidth}\raggedright
O(С*n*log\^{}2(n)) или O(С*n*sqrt(n)), где С=10
\end{minipage} \\
\begin{minipage}[b]{\linewidth}\raggedright
до 3e5
\end{minipage} & \begin{minipage}[b]{\linewidth}\raggedright
То же самое, но без O(n*log\^{}2(n))
\end{minipage} & \begin{minipage}[b]{\linewidth}\raggedright
То же самое, только С меньше C=3
\end{minipage} \\
\begin{minipage}[b]{\linewidth}\raggedright
до 1e6
\end{minipage} & \begin{minipage}[b]{\linewidth}\raggedright
O(n), O(n*log(n)),
\end{minipage} & \begin{minipage}[b]{\linewidth}\raggedright
O(C*n*log(n)), C=5
\end{minipage} \\
\midrule()
\endhead
\bottomrule()
\end{longtable}

\begin{enumerate}
\def\labelenumi{\arabic{enumi})}
\item
  Не забывайте, что у вас может появиться бесконечный цикл. Тут
  ускорения не помогут :)
\item
  ``Программа ждет ввода данных, хотя входной поток уже закончился'' -
  памятка участника. Не думаю, что это у вас будет)
\item
  Ускорение потоков ввода и вывода. Решения были запущены на одном и том
  же коде, вводящем 1e5 чисел и выводящем их, с добавлением
  (ios\_base::sync\_with\_stdio(0), cin.tie(0) или cout.tie(0))
\end{enumerate}

\includegraphics[width=6.26772in,height=0.93056in]{vertopal_dff5ae5125234002b8902efb366dc180/media/image1.png}

Видно, что cout.tie(0) бесполезно. ВЫУЧИТЕ другие два ускорения, ведь
они в 3 раза улучшают время, и не каждая среда может подсказать.

\begin{enumerate}
\def\labelenumi{\arabic{enumi})}
\setcounter{enumi}{3}
\item
  Правильно расставляйте типы переменных, т. к. int занимает меньше
  времени на создание.
  \includegraphics[width=6.26772in,height=1.41667in]{vertopal_dff5ae5125234002b8902efb366dc180/media/image3.png}
\item
  Сделать статические массивы вместо векторов. При этом в местах, где
  vector всё же используется как стек, перед использованием сделать
  v.reserve(MAXN).
\item
  Заменить map и set на unordered\_map и unordered\_set. Не забудьте
  сделать mp.reserve(MAXN), иначе будет долго работать!
\end{enumerate}

Бывают ситуации, когда нужно ключом unordered\_map сделать структуру.
Для этого нужно структуру захешировать и записывать в unordered\_map
хэш.

\begin{enumerate}
\def\labelenumi{\arabic{enumi})}
\setcounter{enumi}{6}
\item
  Прагмы.
\end{enumerate}

Стандартный набор, который я использую во всех решениях:

\begin{longtable}[]{@{}
  >{\raggedright\arraybackslash}p{(\columnwidth - 0\tabcolsep) * \real{1.0000}}@{}}
\toprule()
\begin{minipage}[b]{\linewidth}\raggedright
\begin{quote}
\#pragma GCC optimize("O3,unroll-loops")

// \#pragma GCC target("avx2,bmi,bmi2,lzcnt,popcnt")
\end{quote}
\end{minipage} \\
\midrule()
\endhead
\bottomrule()
\end{longtable}

Заметим, что вторая строка закомментирована. Если не заходит без неё,
можно попробовать её раскомментировать и заслать. Но по умолчанию
отправляйте без неё.

Почитать, за что каждая из них отвечает:
\href{https://codeforces.com/blog/entry/129283?locale=ru}{\uline{https://codeforces.com/blog/entry/129283?locale=ru}}.

Пытаться загнать прагмы только тогда, когда вам нечего больше делать!

\begin{enumerate}
\def\labelenumi{\arabic{enumi})}
\setcounter{enumi}{7}
\item
  По возможности избавляться от любых делений и остатков по модулю.
  Например, взятие остатка по модулю, когда пишешь хэш, работает дольше,
  чем if сумма больше
  модуля.\includegraphics[width=5.49606in,height=1.70833in]{vertopal_dff5ae5125234002b8902efb366dc180/media/image2.png}
\item
  Путем проб и ошибок стало понятно, что переписывать решение в
  тернарные операторы для ускорения по времени бессмысленно.
\item
  Переход по столбцам а потом по строкам в массиве (j, i) работает
  дольше, чем по строкам, а потом по столбцам (i, j, стандартный обход),
  из-за прыжков по памяти длиной n, вместо 1.
\end{enumerate}

\hypertarget{ux43dux435-ux437ux430ux445ux43eux434ux438ux442-ux43fux43e-ml-memory-limit}{%
\section{Не заходит по ML (memory
limit):}\label{ux43dux435-ux437ux430ux445ux43eux434ux438ux442-ux43fux43e-ml-memory-limit}}

256 мб - это примерно 6e7 int и в 2 раза меньше long long и double.

\begin{enumerate}
\def\labelenumi{\arabic{enumi})}
\item
  Правильно расставляйте типы переменных, int занимает в 2 раза меньше
  бит, чем long long.
\item
  Большая рекурсия, возможно стоит попытаться заменить на циклы или
  передавать меньше параметров.
\item
  ``Ошибки при работе с указателями в C/C++ также могут
  диагностироваться, как ML'' - памятка участника (Устарело, бывает в
  старых плюсах).
\item
  Unordered\_map, unordered\_set стоит заменить на map, set.
\item
  Добавить к vector, unordered\_map или unordered\_set после момента их
  создания .reserve(MAXN).
\end{enumerate}

\hypertarget{ux43dux435-ux437ux430ux445ux43eux434ux438ux442-ux43fux43e-re-runtime-error}{%
\section{Не заходит по RE (Runtime
error):}\label{ux43dux435-ux437ux430ux445ux43eux434ux438ux442-ux43fux43e-re-runtime-error}}

RE может быть не видно на вашем компьютере, это называется UB (Undefined
Behavior).

\#define \_GLIBCXX\_DEBUG --- отлавливает все RE в STL-структурах. Если
получили RE на 1 тесте, а локально всё правильно, надо написать это в
начало кода (до includов). Также стоить заметить, что эта дебаг может
делать TL, поэтому используйте его только локально (\#ifdef LOCAL).

\begin{enumerate}
\def\labelenumi{\arabic{enumi})}
\item
  Превышение лимита по памяти также может диагностироваться, как "Ошибка
  выполнения" - памятка участника. (Устарело, возможно только на
  informatics.)
\item
  Выход за пределы массива.
\item
  Деление на 0 (причем в double и long double не возникает RE,
  компилятор выведет inf).
\item
  Разыменовывание указателя на end в векторах и сетах.
\item
  Забыли убрать assert, с помощью него можно удобно дебагать и ifать
  тест.
\end{enumerate}

\hypertarget{ux43dux435ux43cux43dux43eux433ux43e-ux43cux443ux434ux440ux43eux441ux442ux438-ux43eux442-ux432ux430ux448ux438ux445-ux43bux44eux431ux438ux43cux44bux445-ux442ux440ux435ux43dux435ux440ux43eux432-ux43cux43eux441ux43aux432ux44b}{%
\section{Немного мудрости от ваших любимых тренеров
Москвы:}\label{ux43dux435ux43cux43dux43eux433ux43e-ux43cux443ux434ux440ux43eux441ux442ux438-ux43eux442-ux432ux430ux448ux438ux445-ux43bux44eux431ux438ux43cux44bux445-ux442ux440ux435ux43dux435ux440ux43eux432-ux43cux43eux441ux43aux432ux44b}}

\begin{enumerate}
\def\labelenumi{\arabic{enumi})}
\item
  Набирайте частички, не в OKах счастье:)
\item
  Не тратьте много времени на одной задаче, придумайте для себя
  стратегию, которая не зависит от сложности задач или аномальных задач
  (С2 на рюкзак 2025 года). Например:
\end{enumerate}

\begin{enumerate}
\def\labelenumi{\arabic{enumi}.}
\item
  Я (Лёха) решал в порядке DABC на то кол-во баллов, что мог, тратил не
  больше 20 минут на дебаг или размышления о задаче (чтение задачи и
  написание кода сюда не входит). Если понимал, что 20 минут прошли, то
  я переходил к следующей задаче по циклу. А в последний час оценивал
  ситуацию и решал, что мне лучше делать, чтобы набрать как можно больше
  баллов.
\item
  Можно сначала решить D и C на халявные подгруппы, а потом уже решать A
  и B на большие баллы. Так не надо будет беспокоиться о том, что
  остался 0 по D.
\end{enumerate}

\begin{enumerate}
\def\labelenumi{\arabic{enumi})}
\setcounter{enumi}{2}
\item
  Хороший рандом (ставьте случайный константный сид, пробуйте разные):
\end{enumerate}

\begin{longtable}[]{@{}
  >{\raggedright\arraybackslash}p{(\columnwidth - 0\tabcolsep) * \real{1.0000}}@{}}
\toprule()
\begin{minipage}[b]{\linewidth}\raggedright
\begin{quote}
mt19937 \textbf{rng}(73);

uniform\_int\_distribution\textless int\textgreater{} uid(0,
int(1e9));\\
int \textbf{rnd}(int n) \{ // случайное целое число от 0 до n - 1
включительно\\
return uid(rng) \% n; // если берете всегда по одному и тому же модулю,
лучше uid создать от этого модуля, тем самым не брать остаток\\
\}\\
\strut \\
ld \textbf{rndd}() \{ // случайное вещественное число от 0 до 1\\
return ld(rand()) / RAND\_MAX;\\
\}\\
\strut \\
vector \textless int\textgreater{} v;\\
shuffle(v.begin(), v.end(), rng); // перемешать все числа
\end{quote}\strut
\end{minipage} \\
\midrule()
\endhead
\bottomrule()
\end{longtable}

\begin{enumerate}
\def\labelenumi{\arabic{enumi})}
\setcounter{enumi}{3}
\item
  Сводите задачи к более мелким и решайте их по отдельности.
\item
  Если вы пишете ДО, геому или любой другой алгоритм, создавайте удобную
  структуру или шаблон, который вы умеете по памяти безошибочно писать,
  чтобы не тратить много время на дебаг уже известного алгоритма.
\item
  Сppreference поможет, если забыли, как что-то работает. Попробуйте им
  пользоваться дома и на пробном туре.
\end{enumerate}

\hypertarget{ux435ux441ux43bux438-ux438ux434ux435ux438-ux43dux435-ux43fux440ux438ux445ux43eux434ux44fux442-ux438-ux441ux438ux434ux438ux442ux435-ux432ux43fux443ux441ux442ux443ux44e-ux43dux435-ux437ux43dux430ux435ux442ux435-ux447ux442ux43e-ux434ux435ux43bux430ux442ux44c}{%
\section{Если идеи не приходят (и сидите впустую, не знаете, что
делать):}\label{ux435ux441ux43bux438-ux438ux434ux435ux438-ux43dux435-ux43fux440ux438ux445ux43eux434ux44fux442-ux438-ux441ux438ux434ux438ux442ux435-ux432ux43fux443ux441ux442ux443ux44e-ux43dux435-ux437ux43dux430ux435ux442ux435-ux447ux442ux43e-ux434ux435ux43bux430ux442ux44c}}

\begin{enumerate}
\def\labelenumi{\arabic{enumi})}
\item
  Если в условии дан граф, иногда бывает полезно запускать DFS от
  случайной вершины и сделать shuffle списков смежности. Это иногда
  позволяет немного уменьшить время работы (т.к. жюри чаще всего делает
  тесты под запуск от корня 1).
\end{enumerate}

То же самое можно сказать про любую задачу, где структуру входных данных
можно сделать более случайной.

\begin{enumerate}
\def\labelenumi{\arabic{enumi})}
\setcounter{enumi}{1}
\item
  Бывает полезно взять очень много жадников и рандомных решений и
  вывести наилучший из ответов, которые они получили.
\item
  Можно группы с маленьким n ифать и запускать на них переборное
  решение, а на больших группах запускать идею из 5 пункта.
\end{enumerate}

ОЧЕНЬ ВАЖНО, покакать перед туром, чтобы не обосраться на нем. У тебя
всё получится, мы верим в тебя. (°◡°♡)

Спасибо Антону Ныйкину, Антону Барисову, Марку Семенову, Тимуру Лузгову
и всем прочитавшим файлик за помощь в его создании.

\end{document}
